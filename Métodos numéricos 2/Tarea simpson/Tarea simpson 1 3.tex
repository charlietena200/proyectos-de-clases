\documentclass[a4paper,10pt]{article}
\usepackage[left=2cm,top=2cm,right=2cm,bottom=2cm]{geometry}
\usepackage[utf8]{inputenc}
\usepackage{amsthm}
\usepackage{graphicx}
\graphicspath{ {images/} }


\newcommand{\mN}{{\mathbb N}}
\newcommand{\mZ}{{\mathbb Z}}
\newcommand{\cZ}{{\mathcal Z}}
\newcommand{\fL}{{\mathfrak L}}
\newcommand{\dst}{\displaystyle}

%opening
\title{}
\author{}
\date{}

\begin{document}

\maketitle
Tarea: Regla de Simpson 1/3\\\\

Grupo 2402\\\\
INTEGRANTES:\\\\
FRANCO LONA OSCAR\\\\
PICHARDO RIVAS ALEXIS JAIR\\\\
GALLEGOS TENA CARLOS ALBERTO\\\\
\newpage
2. Realiza los ejercicios que te presento a continuación.\\\\  

Dada la siguiente integral\\\\
$\int_{3}^{6} \frac{x}{4+x^2} dx$\\\\
a) Encuentre su valor exacto, con el método analítico\\\\
$\int_{3}^{6} \frac{x}{4+x^2} dx = \frac{ln(x^2 + 4)}{2} |_{3}^{6}$\\\\
$=\frac{ln(6^2 + 4)}{2} - \frac{ln(3^2 + 4)}{2} = 1.84444 - 1.282474 = 0.561965$\\\\
b) Encuentre su valor aproximado con N = 4, con Simpson $\frac{1}{3}$\\\\
\begin{tabular}{| c | c |}
\hline
x & f(x)\\
\hline
a=3 & 0.230769\\
x1=4 & 0.2\\
x2=4.5 & 0.185567\\
x3=5.5 & 0.160583\\
b=6 & 0.15\\
\hline
\end{tabular}\\\\
N=4\\\\
h= $\frac{6-3}{4} = \frac{3}{4} = 0.75$\\\\
A=$\frac{0.75}{3}[0.230769 + 0.15 + 2(0.185567) + 4(0.2+0.160583)]$\\\\
$= 0.25[2.194235] = 0.548558$\\\\
Por lo tanto, la aproximación por Simpson 1/3 de $\int_{3}^{6} \frac{x}{4+x^2} dx$ es 0.548558.\\\\

\end{document}