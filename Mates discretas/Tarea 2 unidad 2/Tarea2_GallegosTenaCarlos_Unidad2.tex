\documentclass[a4paper,10pt]{article}
\usepackage[left=2cm,top=2cm,right=2cm,bottom=2cm]{geometry}
\usepackage[utf8]{inputenc}
\usepackage{amsthm}
\usepackage{graphicx}
\graphicspath{ {images/} }


\newcommand{\mN}{{\mathbb N}}
\newcommand{\mZ}{{\mathbb Z}}
\newcommand{\cZ}{{\mathcal Z}}
\newcommand{\fL}{{\mathfrak L}}
\newcommand{\dst}{\displaystyle}

%opening
\title{}
\author{}
\date{}

\begin{document}

\maketitle
Carlos Gallegos\\\\
Unidad 2 Tarea 2\\\\
ii) (a$\star$b)$\bigoplus(b\star c) = (a\bigoplus b) \star (a\bigoplus c) = b$\\\\
Para (a$\star$b) sabemos que es a, porque a se relaciona con a y b; y a su vez se cumple que $a\leq b$. Entonces a es la máxima cota superior entre a y b. Para $(b\star c)$ tenemos que b es la máxima cota inferior, porque b se relaciona con b y c; y a su vez se cumple que $b\leq c$. Sustituyendo:\\\\
\centerline{(a$\star$b)$\bigoplus(b\star c) = a \bigoplus b = b$}\\\\
Sabemos que la igualdad nos da b porque es la mínima cota superior entre a y b.\\\\
Por otro lado , para $(a\bigoplus b)$ sabemos que es igual a b. Y de forma similar, para $(a\bigoplus c)$ sabemos que es igual a c, porque se cumple que c es la mínima cota superior para a y c dado que $a\leq c$. Sustituyendo:\\\\
\centerline{$(a\bigoplus b) \star (a\bigoplus c) = b \star c$ = b}\\\\
Podemos decir que nos da b, porque entre b y c, b es la máxima cota inferior.\\\\
Por transitividad, se prueba que (a$\star$b)$\bigoplus(b\star c) = (a\bigoplus b) \star (a\bigoplus c) = b$.

\end{document}