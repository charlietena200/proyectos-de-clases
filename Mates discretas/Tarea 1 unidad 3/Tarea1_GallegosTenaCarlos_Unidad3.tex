\documentclass[a4paper,10pt]{article}
\usepackage[left=2cm,top=2cm,right=2cm,bottom=2cm]{geometry}
\usepackage[utf8]{inputenc}
\usepackage{amsthm}
\usepackage{graphicx}
\graphicspath{ {images/} }

\newcommand{\mN}{{\mathbb N}}
\newcommand{\mZ}{{\mathbb Z}}
\newcommand{\cZ}{{\mathcal Z}}
\newcommand{\fL}{{\mathfrak L}}
\newcommand{\dst}{\displaystyle}

%opening
\title{}
\author{}
\date{}

\begin{document}

\maketitle
Carlos Alberto Gallegos Tena \\\\
Tarea 1\\\\
Para convertir el 23 a base 5 primero dividimos entre 5, nos queda 4 y de residuo 3. Ahora 3 entre 5 y nos queda 3 de residuo. Entonces tenemos que 43 es en base 5 lo que 23 es en base decimal. Lo podemos comprobar teniendo en cuenta que:\\\\
La primera posición es $5^0=1$ y la segunda es $5^1 = 5$. Por ello, si tenemos 43, tenemos 4(5)+3(1)=23.\\\\
Siguiendo la misma lógica, si tenemos el número 431 en base 5, hacemos lo mismo; 4($5^2$)+3(5)+1(1)= 4(25)+15+1= 116.


\end{document}