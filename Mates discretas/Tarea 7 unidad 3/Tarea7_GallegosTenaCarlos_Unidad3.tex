\documentclass[a4paper,10pt]{article}
\usepackage[left=2cm,top=2cm,right=2cm,bottom=2cm]{geometry}
\usepackage[utf8]{inputenc}
\usepackage{amsthm}
\usepackage{graphicx}
\graphicspath{ {images/} }

\newcommand{\mN}{{\mathbb N}}
\newcommand{\mZ}{{\mathbb Z}}
\newcommand{\cZ}{{\mathcal Z}}
\newcommand{\fL}{{\mathfrak L}}
\newcommand{\dst}{\displaystyle}

%opening
\title{}
\author{}
\date{}

\begin{document}

\maketitle
Carlos Alberto Gallegos Tena \\\\
Tarea 7 Mates Discretas\\\\
Para la demostración vamos a hacer primero los or. Hacemos x v y\\\\
\begin{tabular}{ c c c c c}
 x & y & x v y \\ 
 0 & 0 & 0 \\  
 0 & 1 & 1 \\
 1 & 0 & 1 \\
 1 & 1 & 1  
\end{tabular}\\\\
Ahora vamos con x v not y\\
\begin{tabular}{ c c c c c}
 x & y & x v noty \\ 
 0 & 0 & 1 \\  
 0 & 1 & 0 \\
 1 & 0 & 1 \\
 1 & 1 & 1  
\end{tabular}\\\\
Ahora con not x v y\\
\begin{tabular}{ c c c c c}
 x & y & notx v y \\ 
 0 & 0 & 1 \\  
 0 & 1 & 1 \\
 1 & 0 & 0 \\
 1 & 1 & 1  
\end{tabular}\\\\
Ahora hacemos x and y\\
\begin{tabular}{ c c c c c}
x & y & x $\wedge$  y \\ 
 0 & 0 & 0 \\  
 0 & 1 & 0 \\
 1 & 0 & 0 \\
 1 & 1 & 1  
\end{tabular}\\\\
Como ya sabemos, and toma únicamente 1 cuando ambos términos son unos. Es decir, si son 3 términos, entonces and sólo va a ser uno cuando los 3 sean 1. Y en este caso, de (x v y) $\wedge$ (x v noty) $\wedge$ (notx v noty), sólo vamos a tener un uno en x=1 y=1. La tabla queda:\\\\
\begin{tabular}{ c c c c c}
x & y & (x v y) $\wedge$ (x v noty) $\wedge$ (notx v noty)  y \\ 
 0 & 0 & 0 \\  
 0 & 1 & 0 \\
 1 & 0 & 0 \\
 1 & 1 & 1  
\end{tabular}\\\\
Por transitividad de la igualdad, y porque ambas funciones tienen la misma tabla de verdad, podemos decir que:\\\\
(x v y) $\wedge$ (x v noty) $\wedge$ (notx v noty) = x $\wedge$.







\end{document}