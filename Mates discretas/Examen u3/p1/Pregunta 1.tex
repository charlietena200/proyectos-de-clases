\documentclass[a4paper,10pt]{article}
\usepackage[left=2cm,top=2cm,right=2cm,bottom=2cm]{geometry}
\usepackage[utf8]{inputenc}
\usepackage{amsthm}
\usepackage{graphicx}
\graphicspath{ {images/} }

\newcommand{\mN}{{\mathbb N}}
\newcommand{\mZ}{{\mathbb Z}}
\newcommand{\cZ}{{\mathcal Z}}
\newcommand{\fL}{{\mathfrak L}}
\newcommand{\dst}{\displaystyle}

%opening
\title{}
\author{}
\date{}

\begin{document}

\maketitle
Carlos Alberto Gallegos Tena \\\\
Primero hacemos la expresión algebráica normal: \\\\
$(\overline{x} \vee y)   \oplus ((\overline{x \vee y})\wedge(x \odot y)) $\\\\
Primero hacemos $(\overline{x \vee y})$ en maxtérminos y nos queda  $(x \vee \overline{y})\wedge(\overline{x} \vee y)\wedge(\overline{x} \vee \overline{y})$\\\\
Ahora ponemos $(x \odot y)$ en maxtérminos y nos queda $(x \vee \overline{y})\wedge(\overline{x} \vee y)$\\\\
Juntando lo que llevamos nos queda:\\\\
$(\overline{x} \vee y)   \oplus ((\overline{x \vee y})\wedge(x \odot y))  = (\overline{x} \vee y)   \oplus ((x \vee \overline{y})\wedge(\overline{x} \vee y)\wedge(\overline{x} \vee \overline{y})) \wedge (x \vee \overline{y})\wedge(\overline{x} \vee y))$\\\\
Ahora para hacer el maxtérmino de $\oplus$, que sabemos que es $(x\vee y )\wedge (\overline{x} \vee \overline{y}) $ hacemos:\\\\
$((\overline{x} \vee y) \vee ((x \vee \overline{y})\wedge(\overline{x} \vee y)\wedge(\overline{x} \vee \overline{y})) \wedge (x \vee \overline{y})\wedge(\overline{x} \vee y))\wedge(\overline{(\overline{x} \vee y)} \vee \overline{((x \vee \overline{y})\wedge(\overline{x} \vee y)\wedge(\overline{x} \vee \overline{y})) \wedge (x \vee \overline{y})\wedge(\overline{x} \vee y))}$\\\\
Por lo tanto, en una expresión de maxtérminos sería:\\\\
$(\overline{x} \vee y)   \oplus ((\overline{x \vee y})\wedge(x \odot y)) $ = \\\\$((\overline{x} \vee y) \vee ((x \vee \overline{y})\wedge(\overline{x} \vee y)\wedge(\overline{x} \vee \overline{y})) \wedge (x \vee \overline{y})\wedge(\overline{x} \vee y))\wedge(\overline{(\overline{x} \vee y)} \vee \overline{((x \vee \overline{y})\wedge(\overline{x} \vee y)\wedge(\overline{x} \vee \overline{y})) \wedge (x \vee \overline{y})\wedge(\overline{x} \vee y))}$\\








\end{document}