\documentclass[a4paper,10pt]{article}
\usepackage[left=2cm,top=2cm,right=2cm,bottom=2cm]{geometry}
\usepackage[utf8]{inputenc}
\usepackage{amsthm}
\usepackage{graphicx}
\graphicspath{ {images/} }

\newcommand{\mN}{{\mathbb N}}
\newcommand{\mZ}{{\mathbb Z}}
\newcommand{\cZ}{{\mathcal Z}}
\newcommand{\fL}{{\mathfrak L}}
\newcommand{\dst}{\displaystyle}

%opening
\title{}
\author{}
\date{}

\begin{document}

\maketitle
Carlos Alberto Gallegos Tena \\\\
Primero veamos que $x \oplus y = (x\vee y) \wedge (\overline{x} \vee \overline{y})$, entonces  $\overline{x \oplus y} = \overline{(x\vee y) \wedge (\overline{x} \vee \overline{y})}$ y por leyes sabemos que \\\\
$=(\overline{x} \wedge \overline{y}) \vee(x\wedge y) $\\\\
Podemos distribuir y nos queda $ = (x \vee \overline{y})\wedge (\overline{x} \vee y)$. Y si nos damos cuenta, es la función de maxtérminos de $(x \odot y) = (x \vee \overline{y})\wedge (\overline{x} \vee y)$.\\\\
Por transitividad : $\overline{x \oplus y} = (x \odot y)$.\\









\end{document}