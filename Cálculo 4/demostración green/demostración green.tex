\documentclass[a4paper,10pt]{article}
\usepackage[left=2cm,top=2cm,right=2cm,bottom=2cm]{geometry}
\usepackage[utf8]{inputenc}
\usepackage{amsthm}
\usepackage{graphicx}
\graphicspath{ {images/} }


\newcommand{\mN}{{\mathbb N}}
\newcommand{\mZ}{{\mathbb Z}}
\newcommand{\cZ}{{\mathcal Z}}
\newcommand{\fL}{{\mathfrak L}}
\newcommand{\dst}{\displaystyle}

%opening
\title{}
\author{}
\date{}

\begin{document}

\maketitle
Carlos Gallegos\\\\
Demostración\\\\
Sabemos por el teorema de las integrales de línea que: \\\\
\centerline{$\oint_C M(x,y) dx = \int_{C1} M(x,y) dx + \int_{C2} M(x,y) dx$}\\\\
Como sabemos que y está en función de x, y dada por g1 y g2, juntando al integral nos queda:\\\\
\centerline{$= \int_a^b M(x,g1(x)) - M(x,g2(x)) dx$}\\\\
Por otro lado, sabemos que:\\\\
\centerline{$\int\int_R \frac{\partial M}{\partial y} dA = \int_a^b M(x,g2(x)) - M(x,g1(x)) dx$}\\\\
Por lo tanto, por transitividad, cambiamos los signos y nos queda:\\\\
\centerline{$\oint_C M(x,y) dx = - \int\int_R \frac{\partial M}{\partial y} dA$}\\\\
Lo mismo ocurre con $\oint_C N(x,y) dy$. Ahora en lugar de usar g1 y g2, usamos su contradominio, que nombraremos h1 y h2. Como ahora estamos respecto a y, nos queda que $\oint_C N(x,y) dy = \int_a^b N(h2(y),y) - N(h1(y),y) dx$\\\\ 
Por ello nos queda $\oint_C N(x,y) dy = \int\int_R \frac{\partial N}{\partial x} dA$\\\\
Al final, simplemente hacemos la suma de integrales y nos queda:\\\\
\centerline{$\oint_C M(x,y) + N(x,y) dy = - \int\int_R \frac{\partial M}{\partial y} dA + \int\int_R \frac{\partial N}{\partial x} dA$}\\\\
Simplificando:\\\\
\centerline{$\int\int_R \frac{\partial N}{\partial x}-\frac{\partial M}{\partial y} dA$}\\\\
Y por lo tanto, queda demostrado el teorema de Green.

\end{document}